\documentclass[oneside,article]{memoir}
\usepackage{layouts}[2001/04/29]
\usepackage{glossaries}

\glstoctrue
\makeglossaries
\makeindex
\def\mychapterstyle{default}
\def\mypagestyle{headings}
\setpnumwidth{2.55em}
\setrmarg{3.55em}
\cftsetindents{part}{0em}{3em}
\cftsetindents{chapter}{0em}{3em}
\cftsetindents{section}{3em}{3em}
\cftsetindents{subsection}{4.5em}{3.9em}
\cftsetindents{subsubsection}{8.4em}{4.8em}
\cftsetindents{paragraph}{10.7em}{5.7em}
\cftsetindents{subparagraph}{12.7em}{6.7em}
\cftsetindents{figure}{0em}{3.0em}
\cftsetindents{table}{0em}{3.0em}
\settrimmedsize{\stockheight}{\stockwidth}{*}
\settrims{0pt}{0pt}
\setlrmarginsandblock{1.5in}{1.5in}{*}
\setulmarginsandblock{1.5in}{1.5in}{*}
\setmarginnotes{17pt}{51pt}{\onelineskip}
\setheadfoot{\onelineskip}{2\onelineskip}
\setheaderspaces{*}{2\onelineskip}{*}
\checkandfixthelayout
\usepackage{fancyvrb}
\usepackage{graphicx}
\usepackage{booktabs}
\usepackage{tabulary}
\usepackage[utf8]{inputenc}
\usepackage[T1]{fontenc}
\usepackage{xcolor}
\usepackage[sort&compress]{natbib}
\VerbatimFootnotes
\def\myauthor{Author}
\def\defaultemail{}
\def\defaultposition{}
\def\defaultdepartment{}
\def\defaultaddress{}
\def\defaultphone{}
\def\defaultfax{}
\def\defaultweb{}
\def\mytitle{Title}
\def\subtitle{}
\def\keywords{}
\def\bibliostyle{plain}
\def\myrecipient{}
\def\latexmode{memoir}
\def\mytitle{Chatty Design Document}
\def\myauthor{Thomas Chiovoloni, Ricky Riggot, Mevludin Guster}
\def\mydate{May 1, 2012}
\bibpunct{(}{)}{;}{n}{}{,}
\usepackage[
plainpages=false,
pdfpagelabels,
pdftitle={\mytitle},
pagebackref,
pdfauthor={\myauthor},
pdfkeywords={\keywords}
]{hyperref}
\usepackage{memhfixc}
\ifx\latexauthor\undefined
\else
\def\myauthor{\latexauthor}
\fi
\ifx\subtitle\undefined
\else
\addtodef{\mytitle}{}{ \\ \subtitle}
\fi
\ifx\affiliation\undefined
\else
\addtodef{\myauthor}{}{ \\ \affiliation}
\fi
\ifx\address\undefined
\else
\addtodef{\myauthor}{}{ \\ \address}
\fi
\ifx\phone\undefined
\else
\addtodef{\myauthor}{}{ \\ \phone}
\fi
\ifx\email\undefined
\else
\addtodef{\myauthor}{}{ \\ \email}
\fi
\ifx\event\undefined
\else
\date[\mydate]{\today}
\fi
\begin{document}
\title{\mytitle}
\author{\myauthor}
\ifx\mydate\undefined
\else
\date{\mydate}
\fi
\mainmatter
\maketitle
\setlength{\parindent}{0pt}
\setlength{\parindent}{1em}

\chapter{Design Overview}
\label{designoverview}

Chatty is a lightweight zero-configuration distributed chat
application. While typically zero-configuration and decentralization
are mutually exclusive, Chatty dodges this by only working on the LAN
(Local Area Network).

Chatty is a UDP-based application, and does not rely on TCP for data
transmission or connection. Every message sent by one of Chatty's
users is broadcast to every IP address on the subnet. To handle the
lost packets associated with UDP, Chatty sends messages each outgoing
message ten times. Because every outgoing messages contains a UUID,
the instance of Chatty receiving these messages is able to correctly
determine and discard duplicates. From our tests we've found that an
individual message has over a 90\% chance of reaching its destination
on a wireless LAN, and well over a 99\% chance (we have never seen it
occur, though it is theoretically possible) of reaching its
destination on a wired LAN. In the case of the wireless LAN, this
still results in a 99.99999999\% of reaching its destination. Should
the guarantee provided prove not to be sufficient, the number of
repetitions is an easily changeable parameter for future developers
who are maintaining Chatty.

\chapter{Explanation of UML Diagram}
\label{explanationofumldiagram}

This section describes the class diagram in \texttt{chattyclassdiagram.pdf},
located in this folder. It was created by running GraphViz on
\texttt{chattyclassdiagram.dot}, also located in this folder.

\section{Overview}
\label{overview}

Chatty follows a Model-View-Controller architecture for its design.

The controller (package \texttt{chatty.controller}) includes the classes
which send and receive UDP data, as well as the classes for managing
the exchange of data between the Model and the View. The controller is
also tasked with abstracting the details of sending and receiving
byte-streams across the network away from the rest of the system.

The model (package \texttt{chatty.model}) contains classes for representing
and managing Users and Messages (both incoming and outgoing), as well
as classes for managing the data which chatty stores while running
(active users, message lists, etc). The \texttt{ChattyModel} class primarily
serves to delegate messages sent to it to their appropriate
destinations.

The view (package \texttt{chatty.view} and subpackage
\texttt{chatty.view.htmleditor}) contains the classes for representing the
model data in a presentable manner. It contains classes for
displaying

\section{Explanation}
\label{explanation}

\subsection{The Controller}
\label{thecontroller}

Package \texttt{chatty.controller} contains six classes, only 5 of which
ended up being used in the final design (\texttt{FileTransferManager} is
present but useless, and regretably was not deleted before the
deadline).

\texttt{ChattyController} is the main entry point of the application, so it
seems like a reasonable place to start. \texttt{ChattyController} does most
of its work during initialization. At this time it starts the \texttt{Inbox}
and \texttt{Outbox}, creates instances of \texttt{ChattyView} and \texttt{ChattyModel}
(telling the view to watch the Model), initializes the \texttt{Settings},
before finally starting to poll the users every second. After
initialization it serves two purposes. First, it organizes the
polling of users every second (as well as telling the model to clear
its timed out users), and delegates messages between portions of the
program which do not need to know details about each other. It is also
the class that organizes the final actions before shutting down,
such as saving the settings, and sending a sign-off notice.

\texttt{Settings} is the simplest class in the controller. It is implemented
as a singleton, and as its name might suggest, it manages the users
settings. Any data which needs to be when the program closes goes
through the settings.

\texttt{ByteMaster}, despite it's humorous name, has a very serious task.
The \texttt{ByteMaster} handles the encoding and decoding of data as bytes
after packets are received and before they are sent. Chatty's current
design does not elegantly handle messages. If it were to be
rearchitected, \texttt{ByteMaster} would be renamed something like
\texttt{MessageFactory}, and would only handle the decoding of messages. The
encoding of messages would be added to part of the \texttt{Message} interface
in the model. The current design of the model will be covered in the
next section, including a discussion of its flaws.

The \texttt{Outbox} manages the sending of messages. It maintains an instance
of a \texttt{Sender}, which is an inner class that implements \texttt{Runnable}, and
runs in a separate thread. There is only ever one instance of
\texttt{Outbox} running at a time, however it is crucial that it not be a
singleton, as it must not be globally available, due to concerns about
thread safety. Its inner class, \texttt{Sender}, maintains a queue of
encoded messages for sending. Five times a second, the sender tries
to send each message in its queue, removing it if it succeeds.

Similar to the \texttt{Outbox}, the \texttt{Inbox} maintains an instance of an inner
class which runs in a separate thread, must only have one instance
(not globally accessible, so not a singleton) open in chatty at a
time. That is where the similarities end, however, as the \texttt{Inbox} is
concerned with the \emph{receiving} of messages. The Inbox has an inner
class, the \texttt{Listener}, which waits on a background thread for packages
sent to Chatty's UDP group. When a message is received, it decodes it
and notifies the \texttt{ChattyController}, who forwards the message to the
appropriate classes. 

\subsection{The Model}
\label{themodel}

Package \texttt{chatty.model} contains classes for representing and managing
domain data. It is fairly small, only containing 5 classes and one
interface, all of which are mostly straightforward. so the discussion
will be briefer than the previous section.

Class \texttt{ChattyModel} is the class representing the entirety of Chatty's
data. It manages instances of the \texttt{UserManager} and \texttt{MessageManager}
classes, and serves primarily to delegate method calls between them
and the rest of Chatty. 

\texttt{MessageManager} manages a list of messages. It implements
\texttt{Observable}, so it can be watched for changes by the view.
\texttt{MessageManager} is also the class which provides the ability to print
messages out to a file (or any instance of OutputStream).

\texttt{Message} is a simple interface defining some getters that any Message
must implement. \texttt{IncomingMessage} and \texttt{OutgoingMessage} both implement
this interface, primarily providing different implementations of
\texttt{toString}.

\texttt{UserManager} manages users, it maintains a set of users, and receives
notification when users perform an action, such as sending us a
message, or responding to pings. It is also the class which handles
performing the action which a message requires be performed
(e.g. change a user's name if a user sends a change-name
message). Similarly to \texttt{MessageManager}, \texttt{UserManager} implements
\texttt{Observable}, so that it may be observed by the view.

The model is the place that suffered the most from our lack of
experience. As you may notice, there are only two classes for
representing messages, however there are several types of messages
being represented. Additionally, you might notice a lack of a \texttt{User}
class. During the initial implementation, no reasonable way could be
found to handle duplicate names. This, combined with flaws in our
original design which lead to a difficult implementation of \texttt{User},
lead to users being represented internally as strings. This was an
enormous mistake, and failed to take advantage of object orientation.
In retrospect, the user class should have maintained three fields,
it's name, it's IP, and a ``last seen'' date. During this hypothetical
redesign, an \texttt{IUser} interface should be created so that in the
future, should change become necessary, we can easily create different
types of users. Additionally, the way messages are handled could use
substantial redesign. The \texttt{Message} interface should remain, but
methods involving serialization to bytes should be added. An
package-private abstract class, \texttt{AMessage}, should implement the
repetitive parts of \texttt{Message} allowing for little overhead for
creation of new message types. The system as a whole should not refer
to \texttt{AMessage}, so not to differentiate between wholly new
implementations of the \texttt{Message} interface and subclasses of
\texttt{AMessage}.\texttt{chatty.controller.ByteMaster}, then would be renamed to
\texttt{chatty.model.MessageFactory}, and would handle the creation of
messages from their underlying byte representations. This design would
greatly simplify the code present throughout the entire system.

\subsection{The View}
\label{theview}

Package \texttt{chatty.view} is conceptually the simplest package in this
program, much of the code inside of it was generated by UI code
generation tools, and so while it contains the bulk of the code
present in the system, the least amount of time will be spent
discussing this package. Primarily it contains 4 classes and one
subpackage, with several auxiliary classes which were used for UI
widgets.

Class \texttt{ChattyView} is created upon initialization of the program by
the controller, and is a subclass of \texttt{JFrame}, and maintains instances
of \texttt{InputPanel}, \texttt{UserPanel}, and \texttt{MessagePanel}, classes for
receiving user input, displaying the list of currently signed-on
users, and displaying the messages. The \texttt{InputPanel} and
\texttt{MessagePanel} are both instances \texttt{HTMLEditor}, from the
\texttt{chatty.view.htmleditor} which is itself a subclass of JEditorPane, so
that it can display rich text, in the form of HTML. The user enters
potentially formatted messages via the \texttt{InputPanel}, which notifies
the \texttt{ChattyView} and in turn the \texttt{ChattyController}. Via the input
panel, the user is able to select several different types of rich text
formatting, which is visible in real time, and applied immediately to
the relevant portion of the \texttt{InputPanel}. Messages travel over the
internet, get received by the view, and are displayed on the
\texttt{MessagePanel} in a formatted manner. 
\if@mainmatter
we're in main
\backmatter
\fi

\end{document}
